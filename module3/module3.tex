\documentclass{article}
\usepackage[utf8]{inputenc}
\usepackage{float}
\usepackage{amsmath}
\usepackage{amssymb}
\usepackage{physics}
\usepackage{bm}
\usepackage{graphicx}

\author{Kandidatnummer: 15010}
\title{FYS3410 - Module III}
\date{April 28, 2018}

\begin{document}
\maketitle

\section*{2. FEFG and }
\emph{Introduce periodic (Born-von Karman) boundary conditions and derive the density of states (DOS) for FEFG in a finite 3D sample. Calculate values of $\varepsilon_F$, $k_F$, $v_F$ and $T_F$, i.e. Fermi energy, wavevector, velocity and temperature, respectively, for alkali metals. Explain the trend.}\\

A FEFG in 3D can at first approximation be described by the free particle Schrödinger equation in three dimensions
\begin{equation}
	-\frac{\hbar^2}{2m} \nabla^2 \bm{\psi_k}(r) = \epsilon_{\bm{k}}\bm{\psi_k}(\bm{r})
	\label{SE}
\end{equation}

Confining the electrons to a cube whose sides have length $L$ we get the following solution for the wavefunctions
\begin{equation}
	\bm{\psi}_n(\bm{r}) = A\sin(\pi n_xx/L)\sin(\pi n_yy/L)\sin(\pi n_zz/L), \quad n_x,n_y,n_z \in \mathbb{Z}
\end{equation}

Next we want to apply Born-von Karman boundary conditions. The Born-von Karman boundary condition can be represented mathematically as certain restrictions on the wavefunction in a crystal under the assumption that the wavefunction is periodic. The condition can be stated as
	\begin{equation}
		\bm{\psi}(x+L,y,z)=\bm{\psi}(x,y,z)
		\label{condition} 
	\end{equation}
For the $x$-direction and similarly for $y$ and $z$.

This is a classical particle in box problem that we know from quantum mechanics. The boundary conditions gives us the familiar solution of
\begin{equation}
	\bm{\psi_k}(\bm{r})e^{i\bm{k}\cdot\bm{r}}
	\label{pwave}
\end{equation}
With the special requirement that the wavevectors are on the form
$$ \bm{k}_x = \frac{2n_x\pi}{L} $$

It is easily shown that this satisfies equation \eqref{condition}
\begin{align*}
	\exp(i\bm{k}\cdot\bm{r}) = \exp(i\bm{k}\cdot(\bm{r}+L\bm{e}_x))\\
	\exp(i\bm{k}\cdot\bm{r}) = \exp(i\bm{k}\cdot\bm{r}) \cdot \exp(iL\bm k_x)\\
	\exp(iL\bm k_x) = 1 \implies Lk_x = 2n_x\pi
\end{align*}

Substituting equation \eqref{pwave} into \eqref{SE} we get an expression for the energies related to each wavevector

\begin{equation}
	\epsilon_{\bm{k}} = \frac{\hbar^2}{2m}\bm{k}^2 = \frac{\hbar^2}{2m}(k_x^2+k_y^2+k_z^2)
\end{equation}

Since energy and wavevector are related in this way, the energy of the possible states in k-space increase radially from the zero-vector in k-space. In the ground state of a system of N free electrons we therefore expect the occupied states to be distributed in a sphere in k-space so as to minimze the energy. The magnitude of the radius is $k_F$ and the related energy is called the Fermi energy

\begin{equation}
	\epsilon_F = \frac{\hbar^2}{2m}k_F^2
\end{equation}

We can now derive an equation for the number of different orbitals, making sure to account for the two possible electrons in each state. We divide the volume of the sphere with the volume that a each state occupies in k-space (inverse distribution density)

\begin{align}
	2*\frac{ 4\pi k_F/3 }{(2\pi/L)^3} = \frac{V}{3\pi^2}k_F^3 = N\\
	N = \frac{V}{3\pi^2}\qty(\frac{2m\epsilon}{\hbar^2})^{3/2}
\end{align}

Finally we now have what we need to find the density of states

\begin{equation}
	D(\epsilon) \equiv \dv{N}{\epsilon} = \frac{V}{2\pi^2}\qty(\frac{2m}{\hbar^2})^{3/2}\sqrt{\epsilon}
\end{equation}

\newpage
Using these relations we can calculate different values for different elements. In the figure are shown for the alkali metals Li, Na, K, Rb and Cs (in order). 

\begin{figure}[H]
	\includegraphics[width = 0.7\linewidth]{4plot.pdf}
	\centering
	\caption{Four plots showing the trends for fermi energy, wavevector, velocity and temperature for the alkali metals}
\end{figure}

\newpage
\section*{3. Temperature dependence of energy and chemical potential for the Fermi-Dirac distribution}
\subsection*{a)}
At zero temperature we have the following expression for Gibbs energy
\begin{equation*}
	G_{T=0}=N\mu=E+pV
\end{equation*}
In the calculations we need to use the expression for the energy
\begin{equation}
	E(N) = \frac{\hbar^2}{2m_e}\qty(\frac{3\pi^2N}{V})^{2/3}
\end{equation}
We want to insert the pressure and we can get this by volume derivating the energy.
\begin{align*}
	p &= -\qty(\pdv{\epsilon(N)}{V})\\
	&=-\pdv{}{V}\qty( \frac{\hbar^2}{2m_e}\qty(\frac{3\pi^2N}{V})^{2/3}  )\\
	&= \frac{\hbar^2}{2m_e}(3\pi^2N)^{2/3} \tfrac{2}{3}V^{-5/3}
\end{align*}
Skipping a few steps of algebra we see that this term looks very similar to the energy and we can abbreviate a lot by introducing this back in. The Gibbs energy becomes
\begin{equation*}
	G = E + \tfrac{2}{3}E = \tfrac{2}{5}E
\end{equation*}

Now we want an expression for the chemical potential where we can insert the result
\begin{align*}
	\mu = \frac{1}{N}(E+pV)\\
	\mu = \frac{5}{3}\frac{E}{N}
\end{align*}

The assignment gave another hint as to how to find E; integrating the product of the density of states, fermi dirac distribution and energy. 
\begin{align*}
	E = \int_0^\infty DOS\cdot FDD\cdot \epsilon \\
	E = \frac{V}{2\pi^2}\qty(\frac{2m_e}{\hbar^2})^{3/2}\frac{2}{5}\epsilon_F^{5/2}
\end{align*}
If we insert this back into the expression for the chemical potential we see that almost everything cancels out and we are left with
\begin{equation}
	\mu = \epsilon_F
\end{equation}

\newpage
\subsection*{b)}

With unitless variables this is what the fermi dirac distribution looks like. It does not match completely but is good approximation.

\begin{figure}[H]
	\includegraphics[width = 0.7\linewidth]{fdd.png}
	\centering
	\caption{The fermi dirac distribution for various temperatures}
\end{figure}

\subsection{c)}


\newpage
\section*{4. DOS in 1D \& 2D FEFG}

\newpage
\section*{ 6. k-space considerations of some cubic lattices }

\end{document}